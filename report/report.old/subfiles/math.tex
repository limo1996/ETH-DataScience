%%%%%%%%%%%%%%%%%%%%%%%%%%%%%%%%%%%%%%%%%%%%%%%%%%%%%%%%%%%%%%%%%%%%%%%%%%%%%%%%
% mathematical packages and shortcut definitions for mathematical an scientific
% reports as well as articles
% created and copyright by Florian Koch - flkoch@student.ethz.ch
%%%%%%%%%%%%%%%%%%%%%%%%%%%%%%%%%%%%%%%%%%%%%%%%%%%%%%%%%%%%%%%%%%%%%%%%%%%%%%%%

%%%%%%%%%%%%%%%%%%%%%%%%%%%%%%%%%%%%%%%%%%%%%%%%%%%%%%%%%%%%%%%%%%%%%%%%%%%%%%%%
% packages that are usefull when setting equations
%%%%%%%%%%%%%%%%%%%%%%%%%%%%%%%%%%%%%%%%%%%%%%%%%%%%%%%%%%%%%%%%%%%%%%%%%%%%%%%%
\usepackage[intlimits,sumlimits]{amsmath}
\usepackage{amsfonts}
\usepackage{amssymb}
\usepackage{amsthm}
\usepackage{mathtools}
\usepackage{graphicx}
\usepackage{commath}
\usepackage{bbm}
\usepackage[Smaller]{cancel}
\usepackage{tikz}
\usetikzlibrary{arrows.meta}
\usepackage{xstring}

\newcommand*{\circled}[1]{\tikz[baseline=(char.base)]{
            \node[shape=circle,draw,inner sep=2pt] (char) {#1};}}
%%%%%%%%%%%%%%%%%%%%%%%%%%%%%%%%%%%%%%%%%%%%%%%%%%%%%%%%%%%%%%%%%%%%%%%%%%%%%%%%
% theorem environments
%\newtheorem{name}[counter]{displayname}[numberwithin]
%%%%%%%%%%%%%%%%%%%%%%%%%%%%%%%%%%%%%%%%%%%%%%%%%%%%%%%%%%%%%%%%%%%%%%%%%%%%%%%%

\theoremstyle{definition}
\newtheorem{theorem}{Theorem}[section]
\newtheorem{proposition}{Proposition}[section]
\newtheorem{lemma}{Lemma}[section]
\newtheorem{corollary}{Corollary}[section]
\newtheorem{conjecture}{Conjecture}[section]
\newtheorem{criterion}{Criterion}[section]
\newtheorem{algorithm}{Algorithm}[section]
\newtheorem{approximation}{Approximation}[section]

\theoremstyle{definition}
\newtheorem{definition}{Definition}[section]
\newtheorem{notation}{Notation}[section]
\newtheorem{condition}{Condition}[section]
\newtheorem{problem}{Problem}[section]
\newtheorem{example}{Example}[section]
\newtheorem{exercise}{Exercise}[section]
\newtheorem{solution}{Solution}[section]

\theoremstyle{remark}
\newtheorem{acknowledgement}{Acknowledgement}[section]
\newtheorem{axiom}{Axiom}[section]
\newtheorem{case}{Case}[section]
\newtheorem{claim}{Claim}[section]
\newtheorem{conclusion}{Conclusion}[section]
\newtheorem{remark}{Remark}[section]
\newtheorem{summary}{Summary}[section]
\newtheorem*{aim}{Aim}

\numberwithin{equation}{chapter} % number equations within sections, or change in needed

%%%%%%%%%%%%%%%%%%%%%%%%%%%%%%%%%%%%%%%%%%%%%%%%%%%%%%%%%%%%%%%%%%%%%%%%%%%%%%%%
% shortcuts sorted in different groups
%%%%%%%%%%%%%%%%%%%%%%%%%%%%%%%%%%%%%%%%%%%%%%%%%%%%%%%%%%%%%%%%%%%%%%%%%%%%%%%%
% standard identifiers for later redefinition
\newcommand{\cedille}[1]{\c{#1}}


% Sets of Numbers
\newcommand{\field}[1]{\mathbb{#1}}
\def\H{\field{H}} % Quaternions
\def\C{\field{C}} % Complex Numbers
\def\Cmz{\ensuremath\field{C}\setminus\{0\}}% Complex Numbers without zero
\def\R{\field{R}} % Real Numbers
\def\Rmz{\ensuremath\field{R}^{-}_{0}}
\def\Rpz{\ensuremath\field{R}^{+}_{0}}
\def\Rm{\ensuremath\field{R}^{-}}
\def\Rp{\ensuremath\field{R}^{+}}
\def\Q{\field{Q}} % Rational Numbers
\def\Qmz{\ensuremath\field{Q}^{-}_{0}}
\def\Qpz{\ensuremath\field{Q}^{+}_{0}}
\def\Qm{\ensuremath\field{Q}^{-}}
\def\Qp{\ensuremath\field{Q}^{+}}
\def\Z{\field{Z}} % Whole Numbers
\def\Zpz{\ensuremath\field{Z}^{+}_{0}}
\def\Zmz{\ensuremath\field{Z}^{-}_{0}}
\def\Zm{\ensuremath\field{Z}^{-}}
\def\Zp{\ensuremath\field{Z}^{+}}
\def\N{\field{N}} % Natural Numbers
\def\Nz{\ensuremath\field{N}_{0}}

% further sets often used
\def\D{\field{D}} % Unit disc



% constants to be set differently
\def\e{\ensuremath{\mathrm{e}}} % Euler's number
\renewcommand{\i}{\ensuremath{i\mkern1mu}} % redefine imaginary unitlength
\renewcommand{\j}{\ensuremath{\mathsf{\bf j}}} % redefine 2nd imaginary unit
\renewcommand{\k}{\ensuremath{\mathsf{\bf k}}} % redefine 3rd imaginary unit



% common functions
%%%%%%%%%%%%%%%%%%%%%%%%%%%%%%%%%%%%%%%%%%%%%%%%%%%%%%%%%%%%%%%%%%%%%%%%%%%%%%%%
% Real and imaginary parts
\renewcommand{\Re}[2][0]{\ensuremath{\operatorname{Re}\!{\del[#1]{#2}}}} % redefine Real part function
\renewcommand{\Im}[2][0]{\ensuremath{\operatorname{Im}\!{\del[#1]{#2}}}} % redefine Imaginary part function


% Numerical
\newcommand{\order}[2][n]{\ensuremath{\mathcal{O}\left(#1^{#2}\right)}} % Order of convergence / remainders

% Matrix functions
\DeclareMathOperator{\curl}{curl} % Rotation of a vectorfield
\DeclareMathOperator{\divo}{div} % Divergence of a vectorfield
\DeclareMathOperator{\grad}{grad} % Gradient of a scalar field
\DeclareMathOperator{\sinc}{sinc} % Sinc function sinc(x)=sin(x)/x

\newcommand{\Curl}[2][-1]{\ensuremath{\curl\!{\del[#1]{#2}}}}
\newcommand{\Div}[2][-1]{\ensuremath{\divo\!{\del[#1]{#2}}}}

\newcommand{\Grad}[2][-1]{\ensuremath{\grad\!{\del[#1]{#2}}}}

\newcommand{\Deg}[2][-1]{\ensuremath{\deg\!{\del[#1]{#2}}}} % define degree function as for polynomials

\newcommand{\Det}[2][-1]{\ensuremath{\det\!{\del[#1]{#2}}}}
\DeclareMathOperator{\tr}{Tr}
\newcommand{\Tr}[2][-1]{\ensuremath{\tr\!{\del[#1]{#2}}}}
\newcommand{\Exp}[2][-1]{\ensuremath{\exp\!{\del[#1]{#2}}}}
\newcommand{\Sin}[2][-1]{\ensuremath{\sin\!{\del[#1]{#2}}}}
\newcommand{\Cos}[2][-1]{\ensuremath{\cos\!{\del[#1]{#2}}}}
\newcommand{\Sinc}[2][-1]{\ensuremath{\sinc\!{\del[#1]{#2}}}}

\newcommand{\Arcsin}[2][-1]{\ensuremath{\arcsin\!{\del[#1]{#2}}}}
\newcommand{\Arccos}[2][-1]{\ensuremath{\cos\!{\del[#1]{#2}}}}
\newcommand{\Sinh}[2][-1]{\ensuremath{\sinh\!{\del[#1]{#2}}}}
\newcommand{\Cosh}[2][-1]{\ensuremath{\cosh\!{\del[#1]{#2}}}}
\newcommand{\Tan}[2][-1]{\ensuremath{\tan\!{\del[#1]{#2}}}}
\newcommand{\Cot}[2][-1]{\ensuremath{\cot\!{\del[#1]{#2}}}}
\newcommand{\Arctan}[2][-1]{\ensuremath{\arctan\!{\del[#1]{#2}}}}
\newcommand{\Tanh}[2][-1]{\ensuremath{\tanh\!{\del[#1]{#2}}}}
\newcommand{\Coth}[2][-1]{\ensuremath{\coth\!{\del[#1]{#2}}}}
\newcommand{\Log}[2][-1]{\ensuremath{\log\!{\del[#1]{#2}}}}
\newcommand{\Lg}[2][-1]{\ensuremath{\lg\!{\del[#1]{#2}}}}

\newcommand{\Ln}[2][-1]{\ensuremath{\ln\!{\del[#1]{#2}}}}



\newcommand{\Sum}[4][-1]{\ensuremath{\sum_{#2}^{#3}{\del[#1]{#2}}}}

% Balls with \B[radius= \epsilon]{center}
\newcommand{\B}[2][\epsilon]{\ensuremath{\operatorname{B}_{#1}(#2)}} % ball
\newcommand{\Bb}[2][\epsilon]{\ensuremath{\partial\!\operatorname{B}_{#1}(#2)}} % boundary of ball
\newcommand{\Bc}[2][\epsilon]{\ensuremath{\overline{\operatorname{B}}_{#1}(#2)}} % closure of ball
\newcommand{\Bt}[3][z_0]{\ensuremath{\operatorname{B}_{#2}^{#3}(#1)}} % annulus
\newcommand{\Btc}[3][z_0]{\ensuremath{\overline{\operatorname{B}}_{#2}^{#3}(#1)}} % closure of the annulus

% derivatives often used
\newcommand{\p}[2][t]{\ensuremath{\pd{#2}{#1}}} % partial \p[varibale= t]{expression}
\newcommand{\px}[1]{\p[x]{#1}} % partial with respect to x
\newcommand{\py}[1]{\p[y]{#1}} % partial with respect to y
\newcommand{\pz}[1]{\p[z]{#1}} % partial with respect to z
\newcommand{\pt}[1]{\p[t]{#1}} % partial with respect to t
\newcommand{\pid}[2][t]{\ensuremath{\partial_{#1}#2}} % partial inline \pid[variable=t]{expression}
\newcommand{\pix}[1]{\pid[x]{#1}} % partial inline with respect to x
\newcommand{\piy}[1]{\pid[y]{#1}} % partial inline with respect to y
\newcommand{\piz}[1]{\pid[z]{#1}} % partial inline with respect to z
\newcommand{\pit}[1]{\pid[t]{#1}} % partial inline with respect to t
\newcommand{\dd}[2][]{\ensuremath{\od{#1}{#2}}} % total derivative \dd{expression}{variabel}
\newcommand{\ddp}[3][]{\ensuremath{\od[#3]{#1}{#2}}} % total higher derivative \ddp{expression}{varibale}{order}
\newcommand{\dt}{\ensuremath{\dif{t}}} % dt
\newcommand{\dx}{\ensuremath{\dif{x}}} % dx
\newcommand{\dy}{\ensuremath{\dif{y}}} % dy
\newcommand{\dz}{\ensuremath{\dif{z}}} % dz
\newcommand{\ddd}{\mathop{}\!\mathrm{d}}
% more shortcuts
\newcommand{\scp}[2]{\ensuremath{\left\langle#1,#2\right\rangle}} % SCalarProduct \scp{first}{second}
\newcommand{\sscp}[2]{\ensuremath{\ #1\cdot#2\ }} % Simple SCalarProduct \sscp{first}{second}
\newcommand{\um}[1]{\ensuremath{\displaystyle\mathbbm{1}_{\!\scalebox{0.5}{\ensuremath{#1}}}}} % UnityMatrix \um{dimension}
\newcommand{\inv}[1][f]{\ensuremath{\frac{1}{#1}}} % INVerse or reciprocal of its argument
\newcommand{\conj}[1]{\ensuremath{\overline{#1}}} % complex CONJugate of its argument
\newcommand{\vect}[1]{\ensuremath{\mathbf{#1}}} % redefinition of vector being printed bold

\newcommand{\degree}{\ensuremath{^{\circ}}}
\newcommand{\comparison}[2][=]{\mathrel{\overset{\makebox[0pt]{\mbox{\normalfont\tiny\sffamily #2}}}{#1}}}
\newcommand{\equal}[1]{\comparison{#1}}
\renewcommand{\c}{\ensuremath{\mathrm{const.}}}


% Arrow shortcuts
\newcommand{\ra}{\ensuremath{\rightarrow}}
\newcommand{\Ra}{\ensuremath{\Rightarrow}}
\newcommand{\lora}{\ensuremath{\longrightarrow}}
\newcommand{\Lora}{\ensuremath{\Longrightarrow}}
\newcommand{\lra}{\ensuremath{\leftrightarrow}}
\newcommand{\Lra}{\ensuremath{\Leftrightarrow}}
\newcommand{\Llra}{\ensuremath{\Longleftrightarrow}}




% Symbols
\DeclareMathSymbol{\mlq}{\mathord}{operators}{``}
\DeclareMathSymbol{\mrq}{\mathord}{operators}{`'}

% Custom
%\newcommand{\labview}{LabView\ensuremath{^{\mathrm{TM}}}}
